\chapter{Introduction}

\section{What Zelig and R Do}

Zelig\footnote{Zelig is named after a Woody Allen movie about a man
  who had the strange ability to become the physical reflection of
  anyone he met --- Scottish, African-American, Indian, Chinese, thin,
  obese, medical doctor, Hassidic rabbi, anything --- and thus to fit
  well in any situation.} is an easy-to-use program that can estimate
and help interpret the results of an enormous and growing range of
statistical models.  It literally \emph{is} ``everyone's statistical
software'' because Zelig's unified framework incorporates everyone
else's (R) code.  We also hope it will \emph{become} ``everyone's
statistical software'' for applications, and we have designed Zelig so
that anyone can use it or add their models to it.

When you are using Zelig, you are also using
\hlink{R}{http://www.r-project.com}, a powerful statistical software
language.  You do not need to learn R separately, however, since this
manual introduces you to R through Zelig, which simplifies
R and reduces the amount of programming knowledge you need to
get started.  Because so many individuals contribute different
packages to R (each with their own syntax and documentation),
estimating a statistical model can be a frustrating experience.  Users
need to know which package contains the model, find the modeling
command within the package, and refer to the manual page for the
model-specific arguments.  In contrast, Zelig users can skip these
start-up costs and move directly to data analyses.  Using Zelig's
unified command syntax, you gain the convenience of a packaged
program, without losing any of the power of R's underlying statistical
procedures.

In addition to generalizing R packages and making existing methods
easier to use, Zelig includes infrastructure that can improve all
existing methods and R programs.  Even if you know R, using Zelig
greatly simplifies your work.  It mimics the popular
\hlink{Clarify}{http://gking.harvard.edu/stats.shtml\#clarify} program
for Stata (and thus the suggestions of \hlink{King, Tomz, and
  Wittenberg,
  2000}{http://gking.harvard.edu/files/abs/making-abs.shtml})
\nocite{KinTomWit00} by translating the raw output of existing statistical
procedures into quantities that are of direct
interest to researchers.  Instead of trying to interpret coefficients
parameterized for modeling convenience, Zelig makes it easy to compute
quantities of real interest: probabilities, predicted values, expected
values, first differences, and risk ratios, along with confidence
intervals, standard errors, or full posterior (or sampling) densities
for all quantities.  Zelig extends Clarify by seamlessly integrating
an option for bootstrapping into the simulation of quantities of
interest.  It also integrates a full suite of nonparametric matching
methods as a preprocessing step to improve the performance of any
parametric model for causal inference (see
\hlink{MatchIt}{http://gking.harvard.edu/matchit}).  For missing data,
Zelig accepts multiply imputed datasets created by
\hlink{Amelia}{http://gking.harvard.edu/stats.shtml\#amelia} (see
\hlink{King, Honaker, Joseph, and Scheve,
  2001}{http://gking.harvard.edu/files/abs/evil-abs.shtml})\nocite{KinHonJos01}
and other programs, allowing users to analyze them as if they were a
single, fully observed dataset.  Zelig outputs replication data sets
so that you (and if you wish, anyone else) will always be able to
replicate the results of your analyses (see \hlink{King,
  1995}{http://gking.harvard.edu/files/abs/replication-abs.shtml}).\nocite{King95}
Several powerful Zelig commands also make running multiple analyses
and recoding variables simple.

Using R in combination with Zelig has several advantages over
commercial statistical software.  R and Zelig are part of the open
source movement, which is roughly based on the principles of science.
That is, anyone who adds functionality to open source software or
wishes to redistribute it (legally) must provide the software
accompanied by its source free of charge.\footnote{As specified in the
  \hlink{GNU General License v.  2}{http://www.gnu.org/copyleft}
  \url{http://www.gnu.org/copyleft}.}  If you find a bug in open
source software and post a note to the appropriate mailing list, a
solution you can use will likely be posted quickly by one of the
thousands of people using the program all over the world.  Since you
can see the source code, you might even be able to fix it yourself.
In contrast, if something goes wrong with commercial software, you
have to wait for the programmers at the company to fix it (and
speaking with them is probably out of the question), and wait for a
new version to be released.

We find that Zelig makes students and colleagues more amenable to
using R, since the startup costs are lower, and since the manual and
software are relatively self-contained.  This manual even includes an
appendix devoted to the basics of advanced R programming, although you
will not need it to run most procedures in Zelig.  A large and growing
fraction of the world's quantitative methodologists and statisticians
are moving to R, and the base of programs available for R is quickly
surpassing all alternatives.  In addition to built-in functions, R is
a complete programming language, which allows you to design new
functions to suit your needs.  R has the dual advantage that you do
not need to understand how to program to use it, but if it turns out
that you want to do something more complicated, you do not need to
learn another program.  In addition, methodologists all over the world
add new functions all the time, so if the function you need wasn't
there yesterday, it may be available today.

\section{Getting Help}

You may find documentation for Zelig on-line (and hence must be
on-line to access it).  If you are unable to connect to the Internet,
we recommend that you print the pdf version of this document for your
reference.

If you are on-line, you may access comprehensive help files for Zelig
commands and for each of the models.  For example, load the Zelig
library and then type at the R prompt:
\begin{verbatim}
> help.zelig(command)                # For help with all zelig commands.
> help.zelig(logit)                  # For help with the logit model.  
\end{verbatim}
\label{Rhelp}In addition, {\tt help.zelig()} searches the manual pages 
for R in addition to the Zelig specific pages.  On certain rare
occasions, the name of the help topic in Zelig and in R are identical.
In these cases, {\tt help.zelig()} will return the Zelig help page by
default.  If you wish to access the R help page, you should use {\tt
  help(topic)}.

In addition, built-in examples with sample data and plots are
available for each model.  For example, type {\tt demo(logit)} to view
the demo for the logit model.  Commented code for each model is
available under the examples section of each model reference page.

Please direct inquiries and problems about Zelig to our listserv at
\hlink{zelig@lists.gking.harvard.edu}{mailto:zelig@lists.gking.harvard.edu}.  We
suggest you subscribe to this mailing list while learning and using
Zelig: go to \hlink{\url{http://lists.hmdc.harvard.edu/index.cgi?info=zelig}}{http://lists.hmdc.harvard.edu/index.cgi?info=zelig}.  (You can choose to receive email
in digest form, so that you will never receive more than one message
per day.)  You can also browse or search our
\hlink{archive}{http://lists.hmdc.harvard.edu/lists/zelig/} of
previous messages before posting your query.

\section{How to Cite Zelig}

To cite Zelig as a whole, please reference these two sources:
\begin{verse}
  Kosuke Imai, Gary King, and Olivia Lau. 2007. ``Zelig: Everyone's
  Statistical Software,'' \url{http://GKing.harvard.edu/zelig}.
\end{verse}
\begin{verse}
Imai, Kosuke, Gary King, and Olivia Lau. (2008). ``Toward A Common Framework for Statistical Analysis and Development.'' Journal of Computational and Graphical Statistics, Vol. 17, No. 4 (December), pp. 892-913. 
\end{verse}


To refer to a particular Zelig model, please refer to the ``how to
cite'' portion at the end of each model documentation section.

%%% Local Variables: 
%%% mode: latex
%%% TeX-master: "zelig"
%%% End: 

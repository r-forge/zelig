\section{{\tt tag}: Constrain parameter effects across equations}
\label{tag}

\subsubsection{Description}
Use {\tt tag()} to identify parameters and constrain their effects
across equations in multiple-equation models.  
  
\subsubsection{Syntax}
\begin{verbatim}
tag(x, label)
\end{verbatim}

\subsubsection{Arguments}
\begin{itemize}
\item {\tt x}: the variable to be constrained.
\item {\tt label}: the name that the constrained variable takes.  
\end{itemize}

\subsubsection{Output Values}
While there is no specific output from {\tt tag()} itself, {\tt
parse.formula()} uses {\tt tag()} to identify parameter constraints
across equations, when a model takes more than one systematic
component.  

\subsubsection{Examples}

\subsubsection{See Also}
\begin{itemize}
\item \Sref{ui} for an overview of the multiple-equation user-interface.
\item \Sref{parse.formula} for more examples of acceptable uses for
{\tt tag()} in formulas.  
\end{itemize}

\subsubsection{Contributors}

Kosuke Imai, Gary King, Olivia Lau, and Ferdinand Alimadhi.


%%% Local Variables: 
%%% mode: latex
%%% TeX-master: t
%%% End: 












